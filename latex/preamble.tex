%Generally useful packages
%-----------------------------------------------
\usepackage{graphicx}
\usepackage[margin=0.75in]{geometry}
\usepackage{caption}
\usepackage{subcaption}
\usepackage{pdflscape} % or {lscape}
\usepackage{longtable}
\usepackage{xcolor}
\usepackage{colortbl}


\setcounter{secnumdepth}{3}
\setcounter{section}{4}  % start at 5 as in the numerical results of the paper

\definecolor{BenchHighlight}{rgb}{0.8,0.8,0.9}
\newcommand{\winner}{\cellcolor{BenchHighlight}}

\newcommand{\BenchmarkResults}[3]{
\captionsetup{labelfont=bf}
\begin{figure}[#1]
  \centering
  \caption{\bf Performance profiles for the #3 problem set}
  \label{fig:#2}
  \begin{subfigure}[b]{0.49\textwidth}
      \centering
      {\includegraphics[width=\textwidth]{../results/plots/bench_#2_performance.pdf}}
      \caption{Relative performance profile}
      \label{fig:#2:relative}
  \end{subfigure}
  \hfill
  \begin{subfigure}[b]{0.49\textwidth}
      \centering
      {\includegraphics[width=\textwidth]{../results/plots/bench_#2_time.pdf}}
      \caption{Absolute performance profile}
      \label{fig:#2:absolute}
  \end{subfigure}
  \begin{subfigure}{1\textwidth}
    \centering
    \footnotesize
    \input{../results/tables/bench_#2_sgm.tex}
    \caption{Benchmark timings as shifted geometric mean and failure rates}
  \end{subfigure}
\end{figure}
}


\newcommand{\detailtablecaption}{FOO}  % just creates the command.   Reset below

\newcommand{\BenchmarkDetailTable}[2]{
\captionsetup{labelfont=bf}
\centering
\renewcommand{\detailtablecaption}{\bf Solve times and iteration counts for the #2 problem set}
\begin{landscape}
  \footnotesize
\input{../results/tables/bench_#1_detail_table.tex}
\end{landscape}
}



